\documentclass{scrreprt}
\setcounter{secnumdepth}{5}
\usepackage{graphicx}
\usepackage[utf8]{inputenc}
\usepackage{tikz}
\usepackage{listings}
\usepackage{underscore}
\usepackage[bookmarks=true]{hyperref}
\usepackage{placeins}
\usepackage{caption}
\hypersetup{
    bookmarks=false,    % show bookmarks bar?
    pdftitle={Software Requirement Specification},    % title
    pdfauthor={Yiannis Lazarides},                     % author
    pdfsubject={TeX and LaTeX},                        % subject of the document
    pdfkeywords={TeX, LaTeX, graphics, images}, % list of keywords
    colorlinks=true,       % false: boxed links; true: colored links
    linkcolor=blue,       % color of internal links
    citecolor=black,       % color of links to bibliography
    filecolor=black,        % color of file links
    urlcolor=purple,        % color of external links
    linktoc=page            % only page is linked
}%
\def\myversion{0.1}
\date{}
\usepackage{hyperref}
\begin{document}
    \begin{titlepage}
        \flushright\bfseries\huge
        \vspace*{\stretch{0.4}}
        \rule{\linewidth}{5pt}
        \par
        \vspace{1cm}
        {\Huge MASTER \par TEST \par DOCUMENT \par}
        \vspace{2cm}
        for \\
        \vspace{2cm}
        Personal Dietary Application \\
        \vspace{2cm}
        \LARGE{Version \myversion \\}
        \vspace{2cm}
        by Craig Boucher \\
        Md Tanveer Alamgir \\
        Fan Zou\\
        Osman Momoh \\
        Xin Ma
        \vspace{2cm}
        \rule{\linewidth}{5pt}
        \vspace{\stretch{1}}
    \end{titlepage}

    \tableofcontents

    \chapter{Introduction}
    This master testing document describes the strategies and tools we (Team 4) used to test the Personal Dietary Manager application. This testing document serves as the deliverable document for iteration 3 of the class project in COMP 5541 at Concordia Univeristy. Iteration 3 is the final iteration of of Personal Dietary Manager, and implements database functionality using MySQL. \\ \\
    Associated with this test plan are documents delivered in iteration 1 and iteration 2 of this project. These are, respectively, the Software Requirements Specifications (SRS) and Design Documents (DD). This test plan will refer to these other documents for the purpose of traceability.

    \section{Purpose}
    The purpose of this test plan is to verify the quality and correctness of the Personal Dietary Manager application. Correctness is the state in which the application satisfies the SRS and delivers a product that can accurately track a dietary regimen. Quality is the state in which the application delivers a user-friendly experience that is free of bugs, runs fast, and runs smoothly. \\\\
    As no software project is perfect, any outstanding bugs and issues will also be discussed.

    \section{Scope}
    The scope of this test document are the tools are strategies that are used for testing. These are:
    \begin{itemize}
        \item Unit testing
        \item Integration and Module Testing
        \item Functional Testing
        \item User Interface Testing
    \end{itemize}
    Tests that are outside of the scope of this document are load, configuration and installation, business analysis, network security, and stress testing.


    \section{Document Terminology and Acronyms}

    \begin{tabular}{|l|l|}
        \hline
        \textbf{Abbreviation} & \textbf{Term} \\
        \hline
        MVC & Model View Controller \\
        \hline
        GUI & Graphical User Interface \\
        \hline
        UML & Unified Modeling Language \\
        \hline
        PDA & Personal Dietary Application \\
        \hline
        GRASP & General Responsibility Assignment Software Patterns \\
        \hline
    \end{tabular}

    \section{References}

    \begin{itemize}

        \item Dr. Nora Houari, 2020, ``COMP 5541 Course Notes" including:
        \begin{itemize}
            \item Introduction to testing
            \item Tutorial: Testing
            \item Project Information
            \item Montrealopoly Master Test Plan
        \end{itemize}
        \item Pressman, R.S. (2009). Software Engineering: A Practitioner's Approach, 7Th Edition.

        \item ISO/IEC/IEEE International Standard - Software and systems engineering -- Software testing --Part 3: Test documentation," in ISO/IEC/IEEE 29119-3:2013(E) , vol., no., pp.1-138, 1 Sept. 2013

    \end{itemize}

    \chapter{Evaluation and Mission Test Motivation}

    \section{Background}

    \section{Evaluation Mission}

    \section{Test Motivators}

    \chapter{Outline of Planned Tests}

    This section provides an overview of the four types of tests that we will do: unit testing, integration and module testing, functional testing, and user interface testing. A brief description of each is described. \\ \\
    We frequently distinguish our testing strategies between white- and black-box testing. White-box testing tests the internal structures of an application (e.g. source code, classes, methods). Conversely, black-box testing tests the external functions that the software is supposed to perform, and does not necessarily examine the underlying source code.

    \section{Unit Testing}

    \section{Integration and Module Testing}

    \section{Functional Testing}
    Functional testing is a form a black-box testing in which the application use cases are tested. These use cases can be found both in the SRS and the Project Infromation class notes document. In other words, the functional requirements of the project are tested for correctness.

    \section{User Interface Testing}
    User interface (UI) testing involves testing the various parts of the GUI for correctness and quality. These GUI parts include the text fields, comboboxes, observable lists, list views, table views, buttons, etc. UI testing is a form of black-box testing.

    \chapter{Test Approach}

    \section{Unit Testing}

    \subsection{Methods...}

    \section{Integration and Module Testing}

    \subsection{MVC...}

    \section{Function Testing}

    \subsection{The use case tests...}

    \section{User Interface Testing}

    \subsection{UI, buttons, etc...}

\end{document}